
\section{Data parsing}

\begin{frame}{Need for parsing}
  % 
  \begin{columns}[T]
    %
    \begin{column}{.4\textwidth}
      Imagine that
      \vspace{0.5cm}
      \begin{arrowlist}
        \itemsep8pt
        \item[]<1-> Data files are generated by a third party (no control
          over the format)
        \item[]<2-> \& the data files need pre-processing
          \vspace{0.3cm}
        \item<3-> Regular expressions provide a powerful and concise way
          to perform pattern match/search/replace over the data
      \end{arrowlist}

    \end{column}
    %
    \begin{column}{.6\textwidth}
      \onslide<4->
      \begin{figure}
        \includegraphics[width=6.4cm]{xkcd208.png}
        \caption*{\tiny  \textcopyright Randall Munroe \href{http://xkcd.com/208/}{xkcd.com} \href{http://creativecommons.org/licenses/by-nc/2.5/}{CC BY-NC 2.5}}
      \end{figure}
    \end{column}    

    %
  \end{columns}
  %
\end{frame}


\begin{frame}[fragile]
  \frametitle{Regular expressions - A case study}
  Formatting street names
  \vspace{0.4cm}
  \begin{minted}{python}
    >>> s = '100 NORTH MAIN ROAD'
  \end{minted}
  \pause
  \vspace{-10pt}
  \begin{minted}{python} 
    >>> s.replace('ROAD', 'RD.')
  \end{minted}
  \pause
  \vspace{-10pt}
  \begin{minted}{python} 
    '100 NORTH MAIN RD.'
  \end{minted}
  \pause
  \vspace{-10pt}
  \begin{minted}{python} 
    >>> s = '100 NORTH BROAD ROAD'
  \end{minted}
  \pause
  \vspace{-10pt}
  \begin{minted}{python}
    >>> s.replace('ROAD', 'RD.') 
  \end{minted}
  \pause
  \vspace{-10pt}
  \begin{minted}{python}
    '100 NORTH BRD. RD.'
  \end{minted}
  \pause
  \vspace{-10pt}
  \begin{minted}{python} 
    >>> s[:-4] + s[-4:].replace('ROAD', 'RD.') 
    '100 NORTH BROAD RD.'
  \end{minted}

  \vspace{0.3cm}
  Better use regular expressions!

  \begin{minted}{python}
    >>> import re 
    >>> re.sub(r'ROAD$', 'RD.', s) 
    '100 NORTH BROAD RD.'
  \end{minted}

  \source{example from Dive Into Python 3 \\ \textcopyright Mark Pilgrim 
    \href{http://creativecommons.org/licenses/by-sa/3.0/}{CC BY-SA 3.0}}

\end{frame}

\begin{frame}{Pattern matching with regular expressions}
\begin{adjustwidth}{2.5em}{0pt}
  \begin{mydescription}{wbbasdss}
    \itemsep3pt
    \item[\^]  Matches beginning of line/pattern
    \item[\$]  Matches end of line/pattern
    \item[. ]  Matches any character except newline
    \item[{[}..{]}]   Matches any single character in brackets
    \item[{[}\^..{]}]  Matches any single character not in brackets
    \item[re*]  Matches 0 or more occurrences of the preceding expression
    \item[re+]  Matches 1 or more occurrences of the preceding expression
    \item[re?]   Matches 0 or 1 occurrence
    \item[re\{n\}]  Match exactly n occurrences
    \item[re\{n,\}]   Match n or more occurrences
    \item[re\{n,m\}]   Match at least n and at most m
  \end{mydescription}
\end{adjustwidth}
  \vspace{0.3cm}
  \pause
  \begin{arrowlist}
    \item Use cheatsheets, trainers, tutorials, builders, etc..
  \end{arrowlist}
\end{frame}


\begin{frame}[fragile]{re.search() \& matches}

  \begin{minted}{python}
    >>> import re
    >>> data = "I like python"
    >>> m = re.search(r'python',data)
  \end{minted}
  \pause
  \vspace{-10pt}
  \begin{minted}{python}
    >>> print m
    <_sre.SRE_Match object at 0x...>
  \end{minted}
  \pause
  \vspace{0.7cm}
  Important properties of the match object:
  
  \begin{adjustwidth}{0.6cm}{0pt}
    \medskip
    \begin{mydescription}{wbasssss}
      \itemsep6pt
      \item[group()] Return the string matched by the RE
      \item[start()] Return the starting position of the match
      \item[end()] Return the ending position of the match
      \item[span()] Return a tuple containing the (start, end) positions of the match
    \end{mydescription}
  \end{adjustwidth}
  
\end{frame}


\begin{frame}[fragile]{re.search() \& matches}

  For example:

  \bigskip

  \begin{minted}{python}
    >>> import re
    >>> data = "I like python"
    >>> m = re.search(r'python',data)
  \end{minted}
  \pause
  \vspace{-10pt}
  \begin{minted}{python}
    >>> m.group()
    'python'
  \end{minted}
  \pause
  \vspace{-10pt}
  \begin{minted}{python}  
    >>> m.start()
    7
  \end{minted}
  \pause
  \vspace{-10pt}
  \begin{minted}{python} 
    >>> m.span()
    (7,13)
  \end{minted}

  \bigskip

  For a complete list of match object properties see for example the
  Python Documentation:

  \smallskip  

  \small \ULurl{https://docs.python.org/2/library/re.html#match-objects}

\end{frame}


\begin{frame}[fragile]{re.findall()}

  \begin{minted}{python}
    >>> import re
    >>> data = "Python is great. I like python"
    >>> m = re.search(r'[pP]ython',data)
  \end{minted}
  \pause
  \vspace{-10pt}
  \begin{minted}{python} 
    >>> m.group()
    'Python'
  \end{minted}

  \bigskip
  \pause
  \begin{arrowlist}
    \item  \textbf{re.search()} returns only the first match, use
      \textbf{re.findall()} instead:
  \end{arrowlist}
  \bigskip
  \pause

  \begin{minted}{python}
    >>> import re
    >>> data = "Python is great. I like python"
    >>> l = re.findall(r'[pP]ython',data)
  \end{minted}
  \pause
  \vspace{-10pt}
  \begin{minted}{python} 
    >>> print l
    ['Python', 'python']
  \end{minted}

  \bigskip
  \pause
  \begin{arrowlist}
    \item  Returns list instead of match object!
  \end{arrowlist}


\end{frame}


\begin{frame}[fragile]{re.findall() - Example}

  \begin{mlinepython}
    import re

    with open("history.txt", "rb") as f:
        text = f.read()

    year_dates = re.findall(r'19[0-9]{2}', text)
  \end{mlinepython}
 
\end{frame}

\begin{frame}[fragile]{re.split()}

  Suppose the data stream has well-defined delimiter

  \vspace{0.4cm}
  \begin{minted}{python}
    >>> data = "x = 20"
    >>> re.split(r'=',data)
    ['x ', ' 20']
  \end{minted}
  \vspace{0.2cm}
  \pause
  \begin{minted}{python} 
    >>> data = 'ftp://python.about.com'
    >>> re.split(r':/{1,3}', data)
    ['ftp', 'python.about.com']
  \end{minted}
  \vspace{0.2cm}
  \pause
  \begin{minted}{python}
    >>> data = '25.657'
    >>> re.split(r'\.',data)
    ['25', '657']
  \end{minted}


\end{frame}

%%% Local Variables: 
%%% mode: latex
%%% TeX-master: t
%%% End: 

%%%%%%%%%%%%%%%%%%%%%%%%%%%%%%%%%%
% Felix Hoffmann, October 2014
% adapted from Jyotika Bahuguna October 2012
%%%%%%%%%%%%%%%%%%%%%%%%%%%%%%%%%%


%\documentclass{beamer}
\documentclass[xcolor=table,10pt]{beamer}
\usepackage{latexsym,amssymb,amsfonts,amsmath}
\usepackage{graphicx}
\usepackage{caption}
\usepackage{minted}
\usepackage{hyperref}
\usepackage{changepage} % for adjust width

%for Introduction - Part 1 - ... Orientation
%\useoutertheme[subsection=false, shadow]{miniframes} 
\useinnertheme{default}
\beamertemplatenavigationsymbolsempty

\usefonttheme{serif}
\usepackage{palatino}
\usepackage{eulervm} %additional math 


% Other Palatino font packages, with math
% see also http://tex.stackexchange.com/questions/89610
% and math_fonts.pdf in Latex docs
% --------------------
% \usepackage{pxfonts}
% \usepackage{mathpazo} % add possibly `sc` and `osf` options
% --------------------

\setbeamercolor*{structure}{fg=black}

\usepackage{transparent} % transparent graphics

\usepackage{textcomp}
\usepackage{subcaption}
\graphicspath{{img/}{../}}

\usepackage[table]{xcolor}
\usepackage{tikz}
\usetikzlibrary{arrows,shapes}
\usepackage{color}

%\usemintedstyle{friendly}
%\usemintedstyle{autumn}
%\usemintedstyle{manni}
%\usemintedstyle{tango}

\newminted[mlinepython]{python}{fontsize=\small, linenos,
               		numbersep=11pt,
               		gobble=4,
               		frame=lines,
                        bgcolor=bg,
               		framesep=3mm}    



% http://tex.stackexchange.com/questions/84936/
\usepackage[loadonly]{enumitem} % Enumitem-Beamer Incompatibility! See
                                % http://tex.stackexchange.com/a/52299/4912
\newlist{arrowlist}{itemize}{1}
\setlist[arrowlist]{label=$\Rightarrow$}


\usepackage[english]{babel}
\usepackage[latin1]{inputenc}

\usepackage{xcolor}

\usepackage[normalem]{ulem}
\hypersetup{%
  colorlinks=true,% hyperlinks will be coloured
  urlcolor=blue,
}
\makeatletter
\DeclareUrlCommand\ULurl@@{%
  \def\UrlFont{\ttfamily\color{blue}}%
  \def\UrlLeft{\uline\bgroup}%
  \def\UrlRight{\egroup}}
\def\ULurl@#1{\hyper@linkurl{\ULurl@@{#1}}{#1}}
\DeclareRobustCommand*\ULurl{\hyper@normalise\ULurl@}
\makeatother


\newenvironment{mydescription}[1]                                               
  {\begin{list}{}%
   {\renewcommand\makelabel[1]{\textbf{##1}\hfill}%
   \settowidth\labelwidth{\makelabel{#1}}%
   \setlength\leftmargin{\labelwidth}
   \addtolength\leftmargin{\labelsep}}}
  {\end{list}}

% cite source for content in the frame in bottom right corner
% usage: \source{here my source}
\usepackage[absolute,overlay]{textpos}
\setbeamercolor{framesource}{fg=gray}
\setbeamerfont{framesource}{size=\scriptsize}
\newcommand{\source}[1]{\begin{textblock*}{5cm}(7.7cm,8.9cm)
    \begin{beamercolorbox}[ht=0.5cm,right]{framesource}
        \usebeamerfont{framesource}\usebeamercolor[fg]{framesource} {#1}
    \end{beamercolorbox}
\end{textblock*}}

%%%%%%%%%%%%%%%%%%%%%%% stretching %%%%%%%%%%%%%%%%%%%%%%%%%%%%%%%

% from http://tex.stackexchange.com/questions/148365 
% and https://gist.github.com/navarroj/7789910

\let\svpar\par
\let\svitemize\itemize
\let\svenditemize\enditemize
\let\svitem\item
\def\newpar{\def\par{\svpar\vfill}}
\def\newitem{\def\item{\vfill\svitem}}
\let\svcenter\center
\let\svendcenter\endcenter
\let\svcolumn\column
\let\svendcolumn\endcolumn
\newlength\columnskip
\columnskip 0pt
\def\newcolumn{%
  \renewenvironment{column}[2]%
    {\svcolumn{##1}\setlength{\parskip}{\columnskip}##2}%
    {\svendcolumn\vspace{\columnskip}}}

\newcommand\stretchy{\only<2>{%
  \newpar\def\item{\svitem\newitem}%
  \renewenvironment{itemize}{\svitemize}{\svenditemize\newpar\par}%
  \renewenvironment{center}{\svcenter\newpar}{\svendcenter\newpar}%
  \newcolumn
}}

%%%%%%%%%%%%%%%%%%%%%%%%%%%%%%%%%%%%%%%%%%%%%%%%%%%%%%%%%%%%%%%%%%%%


\title {File operations, data parsing and batch files}
%\subtitle{}

\author[Felix Hoffmann]{Felix Hoffmann} 
\institute[BCF]{Bernstein Center Freiburg}
\date{\today}


\AtBeginSection[]
{
\begin {frame}<beamer>
\frametitle{}
\tableofcontents[currentsection]
\end{frame}
}


%%%%%%%%%%%%%%%%%%%%%%%%%%%%%%%%%%%%%%%%%%%%%%

\begin{document}

\definecolor{bg}{rgb}{0.95,0.95,0.95}
\definecolor{tg}{rgb}{0.35,0.35,0.35}













% search (start, end etc. https://docs.python.org/3/howto/regex.html)
% , findall, finditer, groups, split, sub













\begin{frame}[fragile]{re.sub()}
\begin{itemize}
\item Most powerful of them all.
\item re.sub(pattern, repl, string, max=0) looks for a pattern and replaces with repl
\item Aim is to delete python style comments in the line below:\\

\begin{minted}{python}
data = "2004-959-559 #This is Phone Number"
ans = re.sub(r'#.*','',data)
print ans
'2004-959-559 '
\end{minted}
\small
\item Now delete the hyphens from the above phone number.\\

\begin{minted}{python}
ans1 = re.sub(r'-','',ans)
print ans1
'2004959559 '
\end{minted}
\end{itemize}
\end{frame}





% %%%%%%%%%%%%% What are they? %%%%%%%%%%% 
% \section{Batch files}
% %%%%%%%%%%%%%%%%%%%%%%%%%%%%%%%%%%%%%%%%
% %%%%%%%%%%%%%%%%%%%%
% \begin{frame}[fragile]
% \frametitle{os module}
% \begin{itemize}
% \item Meta scripts that can automate your task
% \item os module in python provides a way of using os dependent functionality
% 	\begin{itemize}
% 	\item os.mkdir() - Creates a directory (like mkdir)
% 	\item os.chmod() - Change the permissions (like chmod)
% 	\item os.rename() - Rename the old file name with the new file name.\	
% 	\item os.listdir() - List the contents of the directory\
% 	\end{itemize}
% \item batch file batch\_file.py to run your script exercise.py for all data files in a directory.\\

% \begin{minted}{python}
% from os import listdir
% files = listdir(".")
% for f in files: 
% 	os.system('python exercise.py f')
% \end{minted}
% \end{itemize}
% \end{frame}

% %%%%%%%%%%%%%%%%%%%%
% \begin{frame}
% \frametitle{Summary}
% \begin{itemize}
% \item Using these libraries, a lot of workflow can be automated.
% \item One doesnt require to learn something separate for pre-processing, processing and automation. Python does it all.
% \end{itemize}

% Suggestion for exercises:
% Test your regex on the file "regextest.gdf" first.\\ It is a reduced version of data files(only 100 data points).\\
% Hence visual inspection can be used easily to check if regex worked correctly.
% \end{frame} 




\end{document}

